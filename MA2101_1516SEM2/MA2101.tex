\PassOptionsToPackage{svgnames}{xcolor}
\documentclass[12pt]{article}



\usepackage[margin=1in]{geometry}  
\usepackage{graphicx}             
\usepackage{amsmath}              
\usepackage{amsfonts}              
\usepackage{framed}               
\usepackage{amssymb}
\usepackage{array}
\usepackage{amsthm}
\usepackage[nottoc]{tocbibind}
\usepackage{bm} 
\usepackage{enumitem}


\DeclareMathOperator{\Tr}{Tr}
 \newcommand{\im}{\mathrm{i}}
  \newcommand{\diff}{\mathrm{d}}
  \newcommand{\col}{\mathrm{Col}}
  \newcommand{\row}{\mathrm{R}}
  \newcommand{\kerne}{\mathrm{Ker}}
  \newcommand{\nul}{\mathrm{Null}}
  \newcommand{\nullity}{\mathrm{nullity}}
  \newcommand{\rank}{\mathrm{rank}}
  \newcommand{\Hom}{\mathrm{Hom}}
  \newcommand{\id}{\mathrm{id}}
  \newcommand{\ima}{\mathrm{Im}}
  \newcommand{\lcm}{\mathrm{lcm}}
  \newcommand{\inv}{^{-1}}
  \newcommand{\str}{^\ast}
  \newcommand\norm[1]{\left\lVert#1\right\rVert}
\setlength{\parindent}{0cm}
\setlength{\parskip}{0em}
\newcommand{\Lim}[1]{\raisebox{0.5ex}{\scalebox{0.8}{$\displaystyle \lim_{#1}\;$}}}
\newtheorem{definition}{Definition}[section]
\newtheorem{theorem}{Theorem}[section]
\newtheorem{notation}{Notation}[section]
\theoremstyle{definition}
\DeclareMathOperator{\arcsec}{arcsec}
\DeclareMathOperator{\arccot}{arccot}
\DeclareMathOperator{\arccsc}{arccsc}
\DeclareMathOperator{\spn}{Span}
\setcounter{tocdepth}{1}
\begin{document}

\title{PYP Answer - MA2101 AY1516Sem2}
\author{Ma Hongqiang}
\maketitle
\begin{enumerate}
  \item %Q1
  \begin{enumerate}
    \item We shall show that $T$ respects addition and scalar multiplication.\\Observe, for $X,Y\in \mathbb{M}_2(\mathbb{R})$ and $c\in \mathbb{R}$,
    \begin{align*}
    T(X+Y)&=A(X+Y)-(X+Y)A =AX+AY-XA-YA\\ 
    &= (AX-XA)+(AY-YA)=T(X)+T(Y) 
    \end{align*}
    and
    \[
T(cX)=A(cX)=cAX = c(AX) = cT(X)
    \]
    \item
    \begin{enumerate}\item Denote the vectors in $\mathcal{B}$ in the same sequence as that in the question, as $b_1,b_2,b_3,b_4$. \\We have
    \[
    [T]_\mathcal{B}=([T(b_1)]_\mathcal{B},[T(b_2)]_\mathcal{B},[T(b_3)]_\mathcal{B},[T(b_4)]_\mathcal{B})
    \]
    Here, for instance,
    \[
T(b_1) = \begin{pmatrix}2&0\\0&5\end{pmatrix}\begin{pmatrix}1&0\\0&0\end{pmatrix}-\begin{pmatrix}1&0\\0&0\end{pmatrix}\begin{pmatrix}2&0\\0&5\end{pmatrix}=\begin{pmatrix}0&0\\0&0\end{pmatrix}=0b_1+0b_2+0b_3+0b_4
    \]
    Therefore, $[T(b_1)]_\mathcal{B} = \begin{pmatrix}0\\0\\0\\0\end{pmatrix}$.\\
    Using the same process, we have
    \[
[T]_\mathcal{B} = \begin{pmatrix}0&0&0&0\\0&-3&0&0\\0&0&3&0\\0&0&0&0\end{pmatrix}
    \]
    \item \begin{align*}\det(T-xI)&=\det([T]_\mathcal{B}-xI)\\
         &= \det \begin{pmatrix}-x&0&0&0\\0&-3-x&0&0\\0&0&3-x&0\\0&0&0&-x\end{pmatrix}\\
         &=x^2(x+3)(x-3)
    \end{align*}
  \end{enumerate}
  \end{enumerate}
  \item \begin{enumerate}
  \item  \begin{align*}
  &T \text{ is injective}\\
  \Leftrightarrow\;\;\;&\kerne(T)=\{0\}.\\
  \Leftrightarrow\;\;\;&T\begin{pmatrix}a\\b\\c\end{pmatrix}=0\text{ has only solution }\begin{pmatrix}0\\0\\0\end{pmatrix}. \\
  \Leftrightarrow\;\;\;&aX+bY+cZ = 0\text{ only has solution }(a,b,c) = (0,0,0)\\
  \Leftrightarrow\;\;\;&X,Y,Z\text{ linearly independent}
  \end{align*}
  \item \begin{enumerate}
\item The rank of $T$ equals to the rank of $[T]_\mathcal{A}^\mathcal{B}$ in (ii), so it has rank 2. Use rank-nullity theorem, nullity = $\dim \mathbb{R}^3-\text{rank} = 3-2 = 1$.
  \item Let $\mathcal{A}=(a_1,a_2,a_3)$ and $\mathcal{B}=(b_1,b_2,b_3,b_4)$. \\We have
  \[
[T]_\mathcal{A}^\mathcal{B} = ([T(a_1)]_B,[T(a_2)]_B,[T(a_3)]_B)
  \]
  For instance,
  \[
T(a_1) = \begin{pmatrix}1&-1\\1&-1\end{pmatrix} = \begin{pmatrix}1&0\\0&-1\end{pmatrix}+(-1)\begin{pmatrix}0&1\\-1&0\end{pmatrix} = b_2+(-1)b_4
  \]
  Therefore, $[T(a_1)]_\mathcal{B} = \begin{pmatrix}0\\1\\0\\-1\end{pmatrix}$.\\
  Using the same process, we have
  \[
[T]_\mathcal{A}^\mathcal{B}=\begin{pmatrix}0&0&0\\1&-2&1\\0&0&0\\-1&1&0\end{pmatrix}
  \]
        \end{enumerate}
\end{enumerate}
\item 
\begin{enumerate}
  \item Since $V$ is finite dimensional, we apply inducion on $k$. The claim is trivial when $k = 1$.\\Now suppose that the claim holds when $k = r$. We proceed to show that the claim is true when $k = r+1$.\\
  From assumption, we have
  \[
\sum_{i=1}^r \alpha_iv_i = 0
  \]
  has only trivial solutions. Then for the case $k = r+1$. Suppose $v_1,\ldots, v_r,v_{r+1}$ is not linearly independent, then, after relabelling
  \[
v_{r+1} = \sum_{i=1}^r \alpha_i v_i
  \], where $\alpha_i\neq 0$ for some $1\leq i\leq r$.\\Apply $T$ on both sides yields
  \[
\lambda_{r+1}w_{r+1}=\sum_{i=1}^r \alpha_i\lambda_iw_i\;\;\;\;\;(1)
  \]
  whereas multiply $\lambda_{r+1}$ on both sides yields
  \[
\lambda_{r+1}w_{r+1}=\sum_{i=1}^r \alpha_i\lambda_{r+1}w_i\;\;\;\;\;(2)
  \]
  $(2)-(1)$, we have
  \[
0 =\sum_{i=1}^r\alpha_i(\lambda_i-\lambda_{r+1})w_i
  \]
  Since by assumption, $w_i, i = 1,\ldots, r$ are independent, we have
  \[
\alpha_i(\lambda_i-\lambda_{r+1})= 0 \text{ for all }i = 1,\ldots, n 
  \]
  and for those $i$, where $\alpha_i\neq 0$, we have
  \[
\lambda_i = \lambda_{r+1}
  \]
  which is a contradiction about the pairwise distinctiveness of $\lambda_i$.Therefore, the claim must be true for $k=r+1$, and the claim for any $k$ is true.
  \item \begin{enumerate}
  \item We first find the eigenvalues. Let $\lambda$ be one eigenvalue, so $Tv = \lambda v$. We have $v^t = \lambda v$, and $v = \lambda v^t$ by taking transpose. It yields,
  \[
v^t = \lambda^2 v^t
  \]
  which implies the eigenvalues are only $\lambda = \pm 1$.\\Solving $Av=\lambda v$ for each $\lambda$, we will have
\[
P = \begin{pmatrix}
0&0&0&0&0&0&0&0&-1\\
0&0&-1&0&0&0&0&-1&0\\
0&-1&0&0&0&-1&0&0&0&\\
0&0&1&0&0&0&0&-1&0\\
0&0&0&0&0&0&-1&0&0&\\
-1&0&0&0&-1&0&0&0&0\\
0&1&0&0&0&-1&0&0&0\\
1&0&0&0&-1&0&0&0&0\\
0&0&0&-1&0&0&0&0&0
\end{pmatrix}
\]
and
\[
D=\begin{pmatrix}
-1&&&&&&&&\\
&-1&&&&&&&\\
&&-1&&&&&&\\
&&&1&&&&&\\
&&&&1&&&&\\
&&&&&1&&&\\
&&&&&&1&&\\
&&&&&&&1&\\
&&&&&&&&1
\end{pmatrix}
\]
\item 
\[
\det(T)=\det(D) = (-1)^3\times 1^6 = -1
\]
\end{enumerate}
  \end{enumerate}
\item
\begin{enumerate}
  \item Denote $B = \{v_1,v_2,v_3\}$. Clearly, $\lambda neq 0$. Let $k = \frac{1}{\lambda}$.\\The claim is then equivalent to: There are at most three distinct value of $k\in\mathbb{C}$ such that $\{kv_1 + p_1, kv_2 + p_2, kv_3 + p_3\}$ fails to be a basis of $V$.\\
  Let us represent $p_i$ in terms of basis $B$.
  \begin{align*}
  p_1 &= a_1v_1 + a_2v_2 + a_3v_3\\
  p_2 &= b_1v_1 + b_2v_2 + b_3v_3\\
  p_3 &= c_1v_1 + c_2v_2 + c_3v_3
  \end{align*}
  Then the claim is equivalent to: There are at most three distinct values of $k\in\mathbb{C}$ such that $\{(k+a_1)v_1+a_2v_2+a_3v_3,b_1v_1+(k+b_2)v_2+b_3v_3,c_1v_1+c_2v_2+(k+c_3)v_3\}$ fails to be a basis for $V$.\\
  The above claim is equivalent to: There are at most three distinct values of $k\in\mathbb{C}$ such that
  \[
  \det\left(\begin{matrix}k+a_1&a_2&a_3 \\ b_1&k+b_2&b_3 \\ c_1&c_2&k+c_3\end{matrix}\right) = 0\;\;\;(\#)
  \]
  The determinant is a polynomial of degree $3$ in $k$, and by Fundamental Theorem of Algebra, it has most three roots. This proves our claim.
  \item Note that 
  \begin{align*}
  p_1&=v_1+v_2\\
  p_2&=v_1+v_3\\
  p_3&=v_2+v_3\\
  \end{align*}
  So,\[
  \begin{array}{ccc}
  a_1=1&a_2 = 1&a_3=0\\
  b_1=1&b_2=0&b_3=1\\
  c_1=0&c_2=1&c_3=1
\end{array}
  \]
  And substituting in the (\#) equation,
  \[
\det\left(\begin{matrix}k+1&1&0 \\ 1&k&1 \\ 0&1&k+1\end{matrix}\right)=0
  \]
  we have $k=-2,-1,1$. So $\lambda = -\frac{1}{2},-1,1$.
  \end{enumerate}
  \item\begin{enumerate}\item
  We shall show that $\Tr(Y^\ast X)$ satisfies sesquilinearity, symmetry and positivity.\\
  \begin{itemize} 
    \item Sesquilinearity \hfill
    \begin{align*}
&\langle a_1X_1+a_2X_2,Y\rangle\\ =&\Tr(Y^\ast (a_1X_1+a_2X_2))\\=&a_1\Tr(Y^\ast X_1)+a_2\Tr(Y^\ast X_2)\\=&a_1\langle X_1,Y\rangle + a_2\langle X_2,Y\rangle
    \end{align*}
     \begin{align*}
&\langle X,b_1Y_1+b_2Y_2\rangle\\ =&\Tr(((b_1Y_1+b_2Y_2)^\ast)X)\\=&\Tr((\overline{b_1}Y_1\str +\overline{b_2}Y_2\str)X) \\=&\overline{b_1}\Tr(Y_1^\ast X)+\overline{b_2}\Tr(Y_2^\ast X)\\=&\overline{b_1}\langle X_1,Y\rangle + \overline{b_2}\langle X_2,Y\rangle
    \end{align*}
    \item Symmetry
    \begin{align*}
    &\overline{\langle Y,X\rangle} \\
    =&\overline{\Tr(X^\ast Y)}\\
    =&\Tr(\overline{X^\ast Y})\\
    =&\Tr(X^t \overline{Y})\\
    =&\Tr((X^t \overline{Y})^t)\\
    =&\Tr(Y^\ast X)\\
    =&\langle X,Y\rangle
    \end{align*}
    \item Positivity 
    \begin{align*}
\langle X,X\rangle &= \Tr(X^\ast X)\\
&=\sum_{k=1}^n\sum_{l=1}^n X_{kl}\overline{X_{kl}}>0 
    \end{align*}
with equality if and only if  $X_{kl}=0$ for all $k,l$ i.e., X is the $\mathbf{0}$ matrix.
  \end{itemize}
  \item Applying Gram Schmidt Process employed with the above inner-product.
  \begin{align*}
  v_1&=A\\
  v_2 &= B-\frac{\langle B,v_1\rangle}{\langle v_1,v_1\rangle}v_1\\
  &= \begin{pmatrix}1&0&0\\i&0&0\\0&-i&-1\end{pmatrix}-\frac{2}{4}v_1\\
  &=\begin{pmatrix}\frac{1}{2}&-\frac{1}{2}i&0\\i&0&\frac{1}{2}i\\0&-i&-\frac{1}{2}\end{pmatrix}
  \end{align*}
  Normalise $v_1$ and $v_2$, we have
  \[
b_1 = \frac{v_1}{\norm{v_1}}=\frac{1}{\sqrt{4}}\frac{1}{2}\begin{pmatrix}1&i&0\\0&0&-1\\0&0&-1\end{pmatrix}
  \]
  and
  \[
b_2 = \frac{v_2}{\norm{v_2}} = \frac{1}{\sqrt{3}}\begin{pmatrix}\frac{1}{2}&-\frac{1}{2}i&0\\i&0&\frac{1}{2}i\\0&-i&-\frac{1}{2}\end{pmatrix}
  \]
\end{enumerate}
\end{enumerate}
\end{document}