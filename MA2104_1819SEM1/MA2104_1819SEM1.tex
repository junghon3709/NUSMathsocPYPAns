\documentclass{article}

\usepackage{mathtools}
\usepackage[utf8]{inputenc}
\usepackage{amsmath}
\usepackage{amssymb}
\usepackage{graphicx}
\usepackage{bm}
\usepackage{enumitem}
\usepackage{textcomp}
\DeclarePairedDelimiter\ceil{\lceil}{\rceil}
\DeclarePairedDelimiter\floor{\lfloor}{\rfloor}

\usepackage[margin=1in]{geometry}
\setlength\parindent{0pt}
\setlength{\parskip}{5px}

\newtheorem{lemma}{Lemma}

\newcommand\at[2]{\left.#1\right|_{#2}}
\newcommand{\p}[2]{\frac{\partial #1}{\partial #2}}
\newcommand{\abs}[1]{\left| #1 \right|}
\newcommand{\paren}[1]{\left(#1\right)}
\newcommand{\brac}[1]{\left[#1\right]}
\newcommand{\R}{\mathbb{R}}

\begin{document}

\subsection*{Q1}

\begin{enumerate}[label=\alph*.]
\item False
\item True
\item True
\item True
\item True
\item False
\end{enumerate}

\subsection*{Q2}

\begin{enumerate}[label=\alph*.]
\item
\[\int_0^1 \int_0^{2-x^2} f(x,y)\ dy\ dx\]
\item 3
\item 4
\item 7
\end{enumerate}

\subsection*{Q3}

\begin{enumerate}[label=\alph*.]
\item
In circular coordinates
\[f(r,\theta,z) = \frac{1}{4\pi} (\sqrt{1-r^2} - z)\]
\item
\[\int_0^{2\pi} \int_0^1 \int_0^{\sqrt{4-r^2}} 2zr\ dz\ dr\ d\theta = \frac{7}{2}\pi\]
\item
Take volume of half a sphere minus a spherical cap, and divide that by 2
\[\frac{1}{2}\paren{\frac{1}{2}\frac{4}{3}\pi(2)^3 - \frac{\pi(2-\sqrt{2})^2}{3}(6-(2-\sqrt{2}))}\]
\end{enumerate}

\subsection*{Q4}

\begin{enumerate}[label=\alph*.]
\item Yes. The potential function
\[\frac{1}{2}x^2 + \frac{1}{2}y^2 + \frac{1}{2}z^2\]
works
\item Let $E$ be the solid bounded by $S$ and $D$. Note that the flux across $D$ is $0$, as $z=0$ in $D$. Hence, the flux across $S$ is the same as the flux across the surface of $E$, and we can apply Divergence Theorem.
\[\nabla \cdot F = 3\]
\[\iint_S F\cdot dA = \iiint_E 3\ dV = \pi\]
\item 20
\end{enumerate}

\subsection*{Q5}

\begin{enumerate}[label=\alph*.]
\item Change coordinates to $u,v,z$, where $u=x+y, v=x-2y$, then the volume has $-1 \leq v \leq 1, 0 \leq u \leq 2, \frac{2}{3}(2u+v) -3 \leq z \leq \frac{2}{3}(2u+v) -1$
\[\p{(u,v)}{(x,y)} = \abs{\begin{pmatrix}
1 & 1 \\ 1 & -2
\end{pmatrix}} = -3\]
\[
\int_{-1}^1 \int_0^2 \int_{\frac{2}{3}(2u+v)-3}^{\frac{2}{3}(2u+v) -1} \frac{1}{3}\ dz\ du\ dv = \frac{8}{3}
\]

\item
\[-1 \leq y \leq 1\]
\[0 \leq z \leq 1-y^2\]
\[0 \leq x \leq 1-z\]
Hence
\[0 \leq x \leq 1-z \leq 1\]
\[0 \leq z \leq 1-x\]
\[y^2 \leq 1-z \implies -\sqrt{1-z} \leq y \leq \sqrt{1-z}\]
Hence
\[\int_{-1}^{1} \int_0^{1-y^2} \int_0^{1-z} f(x,y,z)\ dx\ dz\ dy = \int_0^1 \int_0^{1-x} \int_{-\sqrt{1-z}}^{\sqrt{1-z}} f(x,y,z)\ dy\ dz\ dx\]
\end{enumerate}


\end{document}
