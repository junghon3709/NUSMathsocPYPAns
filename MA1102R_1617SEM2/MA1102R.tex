\PassOptionsToPackage{svgnames}{xcolor}
\documentclass[12pt]{article}



\usepackage[margin=1in]{geometry}  
\usepackage{graphicx}             
\usepackage{amsmath}              
\usepackage{amsfonts}              
\usepackage{framed}   
\usepackage{mathtools}            
\usepackage{amssymb}
\usepackage{array}
\usepackage{amsthm}
\usepackage[nottoc]{tocbibind}
\usepackage{bm}
\usepackage{enumitem}


\DeclareMathOperator{\Tr}{Tr}
 \newcommand{\im}{\mathrm{i}}
  \newcommand{\diff}{\mathrm{d}}
  \newcommand{\col}{\mathrm{Col}}
  \newcommand{\row}{\mathrm{R}}
  \newcommand{\kerne}{\mathrm{Ker}}
  \newcommand{\nul}{\mathrm{Null}}
  \newcommand{\nullity}{\mathrm{nullity}}
  \newcommand{\rank}{\mathrm{rank}}
  \newcommand{\Hom}{\mathrm{Hom}}
  \newcommand{\id}{\mathrm{id}}
  \newcommand{\ima}{\mathrm{Im}}
  \newcommand{\lcm}{\mathrm{lcm}}
  \newcommand{\diag}{\mathrm{diag}}
  \newcommand{\inv}{^{-1}}
  \newcommand{\str}{^\ast}
  \newcommand\norm[1]{\left\lVert#1\right\rVert}
\setlength{\parindent}{0cm}
\setlength{\parskip}{0em}
\newcommand{\Lim}[1]{\raisebox{0.5ex}{\scalebox{0.8}{$\displaystyle \lim_{#1}\;$}}}
\newtheorem{definition}{Definition}[section]
\newtheorem{theorem}{Theorem}[section]
\newtheorem{notation}{Notation}[section]
\theoremstyle{definition}
\DeclareMathOperator{\arcsec}{arcsec}
\DeclareMathOperator{\arccot}{arccot}
\DeclareMathOperator{\arccsc}{arccsc}
\DeclareMathOperator{\spn}{Span}
\setcounter{tocdepth}{1}
\begin{document}

\title{PYP Answer - MA1102R}
\author{Ma Hongqiang}
\maketitle
\begin{enumerate}
  \item\begin{enumerate}
    \item We note that 
    \[
f'(x)=\begin{cases}
e^{x-3}(1-x)&\text{ if }x\leq 3\\
20 - 16 x + 3 x^2&\text{ if }3<x<5
\end{cases}
    \]
    By definition of critical point, we solve $f'(x)=0$, and we have $x=1$ or $x=\frac{10}{3}$. Also, $f'$ does not exists at $x=3$. So the $x$ coordinate of each critical point is $1,3,\frac{10}{3}$.
    \item We note that $f(x)>0$ for all $x\leq 3$. Also,
    \begin{table}[h]
    \centering
    \begin{tabular}{|c|c|c|c|c|c|c|c|c|}
Interval&$(-\infty, 1)$&$1$&$(1,3)$&$3$&$(3,\frac{10}{3})$&$\frac{10}{3}$&$(\frac{10}{3},5)$&5\\\hline
$f$&$\nearrow$&$e^{-2}$&$\searrow$&$-1$&$\searrow$&$-\frac{32}{27}$&$\nearrow$&9\\
\end{tabular}
\end{table}
Therefore, there is no absolute maximum value of $f$, and the minimum value of $f$ is $-\frac{32}{27}$ at $x=\frac{10}{3}$.
    \item We calculate $f''$.
    \[
f''(x)=\begin{cases}
-e^{x-3}x&\text{ if }x\leq 3\\
-16 + 6 x&\text{ if }3<x<5
\end{cases}
    \]
    So $f''(x)>0$ gives $x<0$ and $3<x<5$.
    \item 
    \begin{align*}
    \int_{-\infty}^3|f(x)|\diff x &=\int_{-\infty}^2f(x)\diff x-\int_2^3f(x)\diff x\\
    &=[e^{x-3}(3-x)]^2_{-\infty}+[-e^{x-3}(3-x)]_2^3\\
    &=e^{-1}-0+e^{-1}\\
    &=2e^{-1}
    \end{align*}
  \end{enumerate}
  \item\begin{enumerate}
    \item\begin{enumerate}
      \item Rearranging, we have $10\int\frac{1}{x^2}\diff x = \int (\frac{1}{t^2}-1)\diff t$. Therefore, we have $-10x^{-1}=-t^{-1}-t+c$. Substituting $x=4, t=2$ into the solution, we have $c=0$. So $x=\frac{10t}{1+t^2}$.
      \item $\frac{\diff x}{\diff t}=\frac{-10(t^2-1)}{(t^2+1)^2}$. Therefore, $\frac{\diff x}{\diff t}=0$ gives $t=1$. We can easily check that $x'>0$ for $t\in(0,1)$ and $x'<0$ for $t\in(1,\infty)$. So the maximum distance is $x(1)=5$.
    \end{enumerate}
    \item Since $z=y^{-2}$, $\frac{\diff z}{\diff y}=-2y^-3$. Multiply $\frac{\diff z}{\diff y}$ on both side of the equation, we have $x^2\frac{\diff z}{\diff x}+2xz = -12\ln(x)$. Dividing both size by $x^2$ arrives at the result.\\
    Using formula, we have $P(x)=\int \frac{2}{x}\diff x = 2\ln x$. Then $v(x)=e^{P(x)}=x^2$. And $y^{-2}=z=\frac{1}{x^2}\int -12\ln(x)\diff x=-\frac{12}{x^2}(x\ln x - x + c)$. Subsituting $x=1, y=1$, we have $c=\frac{11}{12}$. So $y=\sqrt{\frac{1}{-\frac{12}{x^2}(x\ln x - x + \frac{11}{12})}}$.
  \end{enumerate}
  \item\begin{enumerate}
    \item Integrating by part, we have $I_n = [(2-\ln x)^nx]_1^{e^2}-\int_1^{e^2} nx(2-\ln x)^{n-1}(-\frac{1}{x})\diff x = nI_{n-1}-2^n$.
    \item $I_0 = \int_1^{e^2} 1\diff x = e^2-1$. $I_1 = e^2-1-2 = e^2-3$. $I_2 = 2(e^2-3)-4 = 2e^2-10$.
    \item $R=1\times 4 + \int_1^{e^2}y\diff x$. Let $u=\ln x$, then $R=4+\int_0^2 (2-u)^2\diff u =4+[-\frac{1}{3}(2-u)^3]_0^2 = \frac{20}{3}$. 
    \item Employ the cylindrical shell method, $V=\int_0^{e^2}2\pi xy\diff x = \int_0^1 2\pi x(4)\diff x + \int_1^{e^2} 2\pi (2-\ln x)^2\diff x = 4\pi + 2\pi I_2 = 4\pi e^2-16\pi$.
  \end{enumerate}
  \item\begin{enumerate}
    \item Let $\epsilon>0$. Choose $\delta = \epsilon\sqrt{a}$. Then $|x-a|<\delta \Rightarrow$
    \begin{align*}
    |\sin\sqrt{x}-\sin\sqrt{a}|&=|2\sin\frac{\sqrt{x}-\sqrt{a}}{2}\cos\frac{\sqrt{x}+\sqrt{a}}{2}|\\
    &\leq |(\sqrt{x}-\sqrt{a})||(1)|\\
    &\leq\delta|\frac{1}{\sqrt{x}+\sqrt{a}}|\\
    &\leq \delta |\frac{1}{\sqrt{a}}|\\
    &\leq \epsilon
    \end{align*}
    \item By Mean Value Theorem, we have, there exists $c\in[0,1102]$, such that $g'(c)=0$. Therefore, $\frac{1}{2}(f(c))^{-\frac{1}{2}}f'(c)f(1102-c)-f(c)^\frac{1}{2}f'(1102-c)=0$. The result follows from rearraging of the previous equation.
    \item\begin{enumerate}
      \item We know that $f(a)=a<\lambda a_(1-\lambda) b<b=f(b)$. Therefore, by intermediate value theorem, there exists $c\in(a,b)$ such that $f(c)=\lambda a+(1-\lambda)b$.
      \item We have $\alpha\in(a,c)$ such that $f'(\alpha)=\frac{f(c)-f(a)}{c-a}=\frac{(1-\lambda)(b-a)}{c-a}$. Simiarly, we have $\beta\in(c,b)$ such that $f'(\beta)=\frac{f(b)-f(c)}{b-c}=\frac{\lambda(b-a)}{b-c}$. Substituting these value into the equation, we have our result.
    \end{enumerate}
  \end{enumerate}
  \item\begin{enumerate}
    \item LHS=$\frac{x^2-6x+9+216+36x}{(x-3)^2}=\left(\frac{x+15}{x-3}\right)^2$=RHS.\\Therefore, arc length $L=\int_4^5 \frac{x+15}{x-3}\diff x =1+18\ln2$.
    \item 
    \begin{align*}
    \int_2^{2017}\frac{1}{[x]^2-[x]}\diff x &=\sum_{i=2}^{2016}\int_i^{i+1}\frac{1}{[x]^2-[x]}\diff x\\
    &=\sum_{i=2}^{2016}\int_i^{i+1}\frac{1}{i^2-i}\diff x\\
    &=\sum_{i=2}^{2016} \frac{1}{i^2-i}\\
    &=\sum_{i=2}^{2016} \left(\frac{1}{i-1}-\frac{1}{i}\right)\\
    &=1-\frac{1}{2016}\\
    &=\frac{2015}{2016}
    \end{align*}
    \item Note,
    \begin{align*}
    \ln (\lim_{x\to 0}(1+\int_{2x}^{4x}\sin(t^2)\diff t)^{\csc(4x^3)})&=\lim_{x\to 0}\ln(1+\int_{2x}^{4x}\sin(t^2)\diff t)^{\csc(4x^3)}\\ 
    &= \lim_{x\to 0}\csc(4x^3)\ln(1+\int_{2x}^{4x}\sin(t^2)\diff t)\\
    &=\lim_{x\to 0}\frac{\ln(1+\int_{2x}^{4x}\sin(t^2)\diff t)}{\sin (4x^3)}\\
    &=\frac{1}{4} \lim_{x\to 0} \frac{4x^{3}}{\sin(4x^{3})}\frac{\ln \left(1+\int_{2x}^{4x} \sin(t^{2})dt\right)}{\int_{2x}^{4x}\sin(t^{2})dt}\frac{\int_{2x}^{4x}\sin(t^{2})dt}{x^{3}}\\
    &=\frac{1}{4}1\times\frac{56}{3}\times 1\\
    &=\frac{14}{3}
    \end{align*}
    Therefore, the required limit is $e^\frac{14}{3}$.
    \item Let $c=\frac{1}{t}\int_0^t f(x)\diff x$. Then,
    \begin{align*}
    \int_0^t \left(f(x) -c\right)^2 \diff x &= \int_0^t f(x)^2\diff x + \int_0^t c^2\diff x -\int_0^t 2cf(x)\diff x \\
    &= \int_0^t f(x)^2\diff x + tc^2 - 2c\int_0^t f(x)\diff x\\
    &= \int_0^t f(x)^2\diff x + tc^2 -2tc^2  \\ 
    &= \int_0^t f(x)^2\diff x - tc^2\\
    &= \int_0^t f(x)^2\diff x -\frac{1}{t}(\int_0^tf(x)\diff x)^2\geq 0
    \end{align*}
    The result follows the last inequality.\\
    We then take $f(x)=\frac{1}{1+x}$. Then subsituting it into the inequality shown, we have $\frac{t}{1+t}\geq t(\ln(1+t))^2$. We then take the square root to get the result.
  \end{enumerate}
\end{enumerate}
\end{document}