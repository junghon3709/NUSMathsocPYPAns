\PassOptionsToPackage{svgnames}{xcolor}
\documentclass[12pt]{article}



\usepackage[margin=1in]{geometry}  
\usepackage{graphicx}             
\usepackage{amsmath}              
\usepackage{amsfonts}              
\usepackage{framed}   
\usepackage{mathtools}            
\usepackage{amssymb}
\usepackage{array}
\usepackage{amsthm}
\usepackage[nottoc]{tocbibind}
\usepackage{bm}
\usepackage{enumitem}
\usepackage{ulem}

\DeclareMathOperator{\sech}{sech}
\DeclareMathOperator{\cosec}{cosec}
\DeclareMathOperator{\Tr}{Tr}
 \newcommand{\im}{\mathrm{i}}
  \newcommand{\diff}{\mathrm{d}}
  \newcommand{\col}{\mathrm{Col}}
  \newcommand{\row}{\mathrm{R}}
  \newcommand{\kerne}{\mathrm{Ker}}
  \newcommand{\nul}{\mathrm{Null}}
  \newcommand{\nullity}{\mathrm{nullity}}
  \newcommand{\rank}{\mathrm{rank}}
  \newcommand{\Hom}{\mathrm{Hom}}
  \newcommand{\id}{\mathrm{id}}
  \newcommand{\ima}{\mathrm{Im}}
  \newcommand{\lcm}{\mathrm{lcm}}
  \newcommand{\diag}{\mathrm{diag}}
  \newcommand{\inv}{^{-1}}
  \newcommand{\str}{^\ast}
  \newcommand\norm[1]{\left\lVert#1\right\rVert}
  \renewcommand{\labelenumii}{\roman{enumii}}
\setlength{\parindent}{0cm}
\setlength{\parskip}{0em}
\newcommand{\Lim}[1]{\raisebox{0.5ex}{\scalebox{0.8}{$\displaystyle \lim_{#1}\;$}}}
\newtheorem{definition}{Definition}[section]
\newtheorem{theorem}{Theorem}[section]
\newtheorem{notation}{Notation}[section]
\theoremstyle{definition}
\DeclareMathOperator{\arcsec}{arcsec}
\DeclareMathOperator{\arccot}{arccot}
\DeclareMathOperator{\arccsc}{arccsc}
\DeclareMathOperator{\spn}{Span}
\setcounter{tocdepth}{1}
\begin{document}

\title{MA1102R AY1718 Sem 1 Answers}
\author{Lim Li}
\maketitle
\begin{enumerate}
  \item \begin{enumerate}[label=(\roman*)]
          \item \[f(x)=(x^3+4x^2+11x+14)e^{-x}\]
		        \begin{align*}
		          f'(x) =& -(x^3+4x^2+11x+14)e^{-x} + (3x^2+8x+11)e^{-x} > 0 \\
		            \iff & (3x^2+8x+11) - (x^3+4x^2+11x+14) > 0 \\
		            \iff & -x^3 -x^2 - 3x - 3 > 0 \\
		            \iff & (-x-1)(x^2+3) > 0 \\
		            \iff & -1 > x
		        \end{align*}
		        $\therefore$ $f$ is increasing on $(-\infty,-1)$ and decreasing on $(-1,\infty)$
		  \item $f(-1)=6e$ is a local maximum. There is no local minimum.
		  \item \[f'(x)=(-x^3-x^2-3x-3)e^{-x}\]
		        \begin{align*}
		          f''(x) =& (-3x^2-2x-3)e^{-x} - (-x^3-x^2-3x-3)e^{-x} \\
		                 =& (x^3 -2x^2 +x)e^{-x} > 0 \\
		             \iff & x^3 -2x^2 + x > 0 \\
		             \iff & x(x-1)^2 > 0 \\
		             \iff & x > 0
		        \end{align*}
		        $f$ is concave up on $(0,\infty)$ and concave down on $(-\infty,0)$
		  \item $f(0)=14$ \\
		        $(0,14)$
        \end{enumerate}
  \item \begin{enumerate}[label=(\alph*)]
          \item For any $\epsilon > 0$, choose $\delta = \min(\epsilon,1)$
                
                Then for all $x$ such that $0<|x-2|<\delta$
                \begin{align*}
                  \left|\frac{x}{x^2+2}-\frac{1}{3}\right| &= \left|\frac{3x-(x^2+2)}{3(x^2+2)}\right|=\left|\frac{(x-1)(x-2)}{3(x^2+2)}\right| \\
                    &< |x-1|\left|\frac{1}{3(x^2+2)}\right|\epsilon && \text{since }|x-2|<\epsilon\\
                    &< 2\times\frac{1}{3\times 2}\times\epsilon && \text{since }|x-1|<2\text{ and }\frac{1}{x^2+2}<\frac{1}{2}\\
                    &< \epsilon
                \end{align*}
          \item \begin{align*}
                  \lim_{n\to \infty} \sum_{i=1}^n \frac{i^3}{n^2(n^2+i^2)} &= \lim_{n\to \infty} \frac{1}{n} \sum_{i=1}^n \frac{(\frac{i}{n})^3}{1+(\frac{i}{n})^2} \\
                    &= \int_0^1 \frac{x^3}{1+x^2}\ dx \\
                    &= \int_0^1 x-\frac{x}{1+x^2}\ dx \\
                    &= \left[\frac{1}{2}x-\frac{1}{2}\ln(x^2+1)\right]_0^1 \\
                    &= \frac{1}{2}-\frac{1}{2}\ln 2
                \end{align*}
          \item \begin{align*}
                  \lim_{x\to 0^+}\left(\frac{e^2-1}{x}\right)^{1/x} &= \lim_{x\to 0^+} \exp\left(\frac{1}{x}\ln\left(\frac{e^x-1}{x}\right) \right) \\
                    &= \lim_{x\to 0^+} \exp\left(\left(\frac{x}{e^x-1}\right)\left(\frac{e^x}{x}-\frac{e^x-1}{x^2} \right) \right) && \text{By L'Hôpital's rule} \\
                    &= \lim_{x\to 0^+} \exp\left(\frac{xe^x-e^x+1}{x(e^x-1)} \right) \\
                    &= \lim_{x\to 0^+} \exp\left(\frac{xe^x}{xe^x+e^x-1)} \right) && \text{By L'Hôpital's rule} \\
                    &= \lim_{x\to 0^+} \exp\left(\frac{e^x}{e^x+\frac{e^x-1}{x}}\right) \\
                    &= \exp\left(\frac{1}{2}\right) \\
                    &= \sqrt{e}
                \end{align*}
        \end{enumerate}
  \item Let the angle of the sector be $\theta$
        \[2r+r\theta=50 \implies \theta = \frac{50}{r} -2\]
        \begin{align*}
          \text{Area} &= \frac{1}{2}r^2\theta \\
            &= \frac{1}{2}r^2\left(\frac{50}{r}-2\right) \\
            &= 25r - r^2 \\
            &= r(25-r) \\
            &\leq \left(\frac{25}{2}\right)^2 && \text{By AMGM inequality, with equality at r=12.5}
        \end{align*}
        $r=12.5$ m
  \item \begin{enumerate}[label=(\alph*)]
          \item \[\ln y = (\sec x)\ln(\tan x) + (\tan x)\ln(\sec x)\]
                \[\frac{1}{y}\frac{dy}{dx} = \sec^2 x\sin x \ln(\tan x) + \cosec x \sec^2 x + \sec^2 x \ln (\sec x) + \sin^2 x\sec^2 x\]
                If $x=\frac{\pi}{4}$, then $y=1^{\sqrt{2}}\sqrt{2}^1=\sqrt{2}$
                
                Sub $x=\frac{\pi}{4}$ and $y=\sqrt{2}$ to the equation
                \[\frac{1}{\sqrt{2}}\frac{dy}{dx} = 0 + 2\sqrt{2} + 2\ln\sqrt{2} +1\]
                \[\frac{dy}{dx} = 4+2\sqrt{2}\ln\sqrt{2}+\sqrt{2}\]
          \item For $x\neq 0$:
                \begin{align*}
                  F'(x) &= \frac{d}{dx}\int_0^{x^2}f(t)\ dt \\
                    &= 2xf(x^2) \\
                    &= \frac{2\sin(x^2)}{x} 
                \end{align*}
                For $x = 0$:
                \[F'(x) = 0\]
                $\therefore$ $F'(x)=0$ for $x=\sqrt{k\pi},k\in \mathbb{Z}$
                
                To check if it is a local max or min, we check the concavity
                
                For $x\neq 0$:
                \[F''(x) = 4\cos(x^2)-\frac{2\sin(x^2)}{x^2}\]
                For $x=0$:
                \[F''(x) = 2\]
                $F''(\sqrt{k\pi} = 4(-1)^k$ for $k\neq 0$ and $F''(0) = 2$
                
                $f$ attains local min at $x=\sqrt{k\pi}$ for even $k$ and local max at $x=\sqrt{k\pi}$ for odd $k$.
          \item \[f''(x)<0 \implies f'(x)\text{ is decreasing} \implies f'(x)<0 \implies f(x)\text{ is decreasing}\]
                Either $\lim_{x\to \infty}f(x)=-\infty$ or $\lim_{x\to \infty}f(x)=k$ for some constant $k$
                
                If $\lim_{x\to \infty}f(x)=k$, then $\lim_{x\to \infty}f'(x)=0$. However, $f'(1)=f'(0)+\int_0^1 f''(x)\ dx<f'(0)=0$. Hence, $\lim_{x\to \infty}f'(x) > f'(1)$, a contradiction.
                
                $\therefore \lim_{x\to \infty}f(x)=-\infty$
                
                $f(x)$ is decreasing and $\lim_{x\to \infty}f(x)=-\infty$ $\implies$ exactly 1 root
        \end{enumerate}
  \item \[y^2 = 2x = 8 - x^2\]
        \[\therefore x=2,y=\pm 2\]
        The curves intersects at $(2,2)$ and $(2,-2)$
        \begin{enumerate}[label=(\roman*)]
          \item \[x^2+y^2=8 \implies x=\sqrt{8-y^2}\]
                \[y^2 = 3x \implies x=\frac{1}{2}y^2\]
                \begin{align*}
                  \text{Area} &= \int_{-2}^2 \sqrt{8-y^2}-\frac{1}{2}y^2\ dy \\
                    &= \int_{-2}^2 \sqrt{8-y^2}\ dy - \left[-\frac{1}{6}y^3\right]_{-2}^2 && \text{sub $y=\sqrt{8}\sin\theta$} \\
                    &= \int_{-\pi/4}^{\pi/4}8\cos^2\theta\ d\theta - \frac{8}{3} \\
                    &= 4\int_{-\pi/4}^{\pi/4} \cos(2\theta)+1\ d\theta - \frac{8}{3} \\
                    &= 2[\sin(2\theta)+2\theta]_{-\pi/4}^{\pi/4} - \frac{8}{3} \\
                    &= \frac{4}{3} + 2\pi 
                \end{align*}
          \item \[x^2+y^2=8 \implies y=\sqrt{8-x^2}\]
                \[y^2 = 3x \implies y=\sqrt{2x}\]
                \begin{align*}
                  \text{Volume} &= 2\left[\int_0^2 \sqrt{2x}(2\pi x)\ dx + \int_2^{\sqrt{8}}\sqrt{8-x^2}(2\pi x)\ dx \right] \\
                    &= 4\pi\left[\left[\frac{2}{5}\sqrt{2}x^{5/2}\right]_0^2 + \left[-\frac{1}{3}(8-x^2)^{3/2} \right]_2^{\sqrt{8}} \right] \\
                    &= 4\pi\left[\frac{16}{5} + \frac{8}{3}\right] \\
                    &= \frac{352}{15}\pi
                \end{align*}
        \end{enumerate}
   \item \begin{enumerate}[label=(\roman*)]
           \item \begin{align*}
                   \int \frac{x\ln x}{(1+x^2)^2}\ dx &= \left(-\frac{1}{2}\right)\frac{\ln x}{1+x^2} - \int\left(-\frac{1}{2}\right)\frac{1}{x(1+x^2)}\ dx && \text{sub }x=\tan\theta\\
                     &= -\frac{\ln x}{2(1+x^2)} + \frac{1}{2}\int\frac{1}{\tan\theta\sec^2\theta}\sec^2\ d\theta \\
                     &= -\frac{\ln x}{2(1+x^2)} + \frac{1}{2}\int\frac{\cos\theta}{\sin\theta}\ d\theta \\
                     &= -\frac{\ln x}{2(1+x^2)} + \frac{1}{2}\ln(\sin\theta)+C \\
                     &= -\frac{\ln x}{2(1+x^2)} + \frac{1}{2}\ln\left(\frac{x}{\sqrt{1+x^2}}\right)+C
                 \end{align*}
           \item \begin{align*}
                   \lim_{x\to \infty}\left(\frac{1}{2}\ln\frac{x}{\sqrt{1+x^2}}-\frac{\ln x}{2(1+x^2)} \right) &= \frac{1}{2}\ln(1) \\
                     &= 0 \\
                   \lim_{x\to 0^+}\left(\frac{1}{2}\ln\frac{x}{\sqrt{1+x^2}}-\frac{\ln x}{2(1+x^2)} \right) &= \lim_{x\to 0^+} \left(\frac{1}{2}\ln x - \frac{\ln x}{2(1+x^2)} - \frac{1}{2}\ln\sqrt{1+x^2} \right) \\
                     &= \lim_{x\to 0^+}\left((\ln x)\left(\frac{1}{2} - \frac{1}{2(1+x^2)}\right) \right) \\
                     &= \frac{1}{2} \lim_{x\to 0^+}\left(\frac{x^2\ln x}{1+x^2}\right) \\
                     &= \frac{1}{2} \lim_{x\to 0^+}(x^2\ln x) \\
                     &= \frac{1}{2} \lim_{x\to 0^+} \frac{1/x}{-2x^{-3}} \hspace{2cm}\text{By L'Hôpital's rule} \\
                     &= 0
                 \end{align*}
                 \[\therefore \int_0^\infty \frac{x\ln x}{(1+x^2)^2}\ dx=0\]
         \end{enumerate}
  \item \begin{enumerate}[label=(\alph*)]
          \item \[y=\frac{1}{x}+\frac{1}{z}\implies z=1\text{ at }x=1\]
                \[\frac{dy}{dx} = -\frac{1}{x^2} - \frac{1}{z^2}\frac{dz}{dx}\]
                \[\frac{dy}{dx} = -\frac{1}{x^2} - \frac{1}{z^2}\frac{dz}{dx} = \left(\frac{1}{x}+\frac{1}{z}\right)^2 - \frac{1}{x}\left(\frac{1}{x}+\frac{1}{z}\right) - \frac{1}{x^2}\]
                \[-\frac{1}{z^2}\frac{dy}{dx} = \frac{1}{z^2}+\frac{1}{xz}\]
                \[\frac{dy}{dx} + \frac{z}{x} + 1 = 0\]
                Let $w=\frac{z}{x}$. Then $w=1$ at $x=1$.
                \[wx=z \implies x\frac{dw}{dx} + w = \frac{dz}{dx}\]
                \[\frac{dz}{dx} = x\frac{dw}{dx} + w = -1-w\]
                \[\int \frac{1}{-1-2w}\ dw = \int\frac{1}{x}\ dx\]
                \[-\frac{1}{2}\ln|1+2w| = \ln(x)+C\]
                Substitute $x=1,w=1$
                \[-\frac{1}{2}\ln 3 = C\]
                Therefore,
                \[-\frac{1}{2}\ln|1+2w|=\ln\frac{x}{\sqrt{3}}\]
                \[\frac{1}{\sqrt{1+2w}}=\frac{x}{\sqrt{3}}\]
                \[w=\frac{1}{2}\left(\frac{3}{x^2}-1\right)\]
                \[z=wx=\frac{3}{2x}-\frac{x}{2} = \frac{3-x^2}{2x}\]
                \[y=\frac{1}{x} + \frac{2x}{3-x^2}\]
          \item \[\int\frac{4h-h^2}{\sqrt{h}}\ dh = \int -1\ dt\]
		        \[\frac{8}{3}h^{3/2} - \frac{2}{5}h^{5/2} + C = -t\]
		        Substitute $t=0,h=4$
		        \[\frac{64}{3}-\frac{64}{5}+C=0\implies C=-\frac{128}{15}\]
		        Therefore,
		        \[\frac{8}{3}h^{3/2} - \frac{2}{5}h^{5/2} -\frac{128}{15} = -t\]
		        Substitute $h=0$
		        \[t=\frac{128}{15}\]
		        $128/15$ minutes
        \end{enumerate}
  \item We first want to show that $\forall x\leq 0.5,|f(x)|\leq Mx$.
  
        Suppose $\exists a\in(0,1)$ such that $|f(a)|>Mx$, then by mean value theorem, $\exists b\in (0,a)$ such that $|f'(b)|=|(f(a)-0)/(a-0)|>M$, a contradiction.
        
        Therefore, $\forall x\leq 0.5,|f(x)|\leq Mx$.
        
        Similarly, we can also show that $\forall x\geq 0.5,|f(x)|\leq M(1-x)$.
        
        Suppose $\exists a\in(0,1)$ such that $|f(a)|>M(1-x)$, then by mean value theorem, $\exists b\in (a,1)$ such that $|f'(b)|=|(f(a)-0)/(a-1)|>M$, a contradiction.
        
        Hence,
        \begin{align*}
          \int_0^1 |f(x)|\ dx &= \int_0^{0.5} |f(x)|\ dx + \int_{0.5}^1 |f(x)|\ dx \\
            &< \int_0^{0.5} Mx\ dx + \int_{0.5}^1 M(1-x)\ dx \\
            &= \frac{1}{4}M
        \end{align*}
        
        
\end{enumerate}
\end{document}