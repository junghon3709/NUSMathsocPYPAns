\PassOptionsToPackage{svgnames}{xcolor}
\documentclass[12pt]{article}



\usepackage[margin=1in]{geometry}  
\usepackage{graphicx}             
\usepackage{amsmath}              
\usepackage{amsfonts}              
\usepackage{framed}   
\usepackage{mathtools}            
\usepackage{amssymb}
\usepackage{array}
\usepackage{amsthm}
\usepackage[nottoc]{tocbibind}
\usepackage{bm}
\usepackage{enumitem}


\DeclareMathOperator{\Tr}{Tr}
 \newcommand{\im}{\mathrm{i}}
  \newcommand{\diff}{\mathrm{d}}
  \newcommand{\col}{\mathrm{Col}}
  \newcommand{\row}{\mathrm{R}}
  \newcommand{\kerne}{\mathrm{Ker}}
  \newcommand{\nul}{\mathrm{Null}}
  \newcommand{\nullity}{\mathrm{nullity}}
  \newcommand{\rank}{\mathrm{rank}}
  \newcommand{\Hom}{\mathrm{Hom}}
  \newcommand{\id}{\mathrm{id}}
  \newcommand{\ima}{\mathrm{Im}}
  \newcommand{\lcm}{\mathrm{lcm}}
  \newcommand{\diag}{\mathrm{diag}}
  \newcommand{\inv}{^{-1}}
  \newcommand{\str}{^\ast}
  \newcommand\norm[1]{\left\lVert#1\right\rVert}
\setlength{\parindent}{0cm}
\setlength{\parskip}{0em}
\newcommand{\Lim}[1]{\raisebox{0.5ex}{\scalebox{0.8}{$\displaystyle \lim_{#1}\;$}}}
\newtheorem{definition}{Definition}[section]
\newtheorem{theorem}{Theorem}[section]
\newtheorem{notation}{Notation}[section]
\theoremstyle{definition}
\DeclareMathOperator{\arcsec}{arcsec}
\DeclareMathOperator{\arccot}{arccot}
\DeclareMathOperator{\arccsc}{arccsc}
\DeclareMathOperator{\spn}{Span}
\setcounter{tocdepth}{1}
\begin{document}

\title{PYP Answer - MA2108 AY1617Sem2}
\author{Ma Hongqiang}
\maketitle
\begin{enumerate}
  \item \begin{enumerate}%Q1
    \item\begin{enumerate}%Q1a
      \item Use squeeze theorem
      \begin{align*}
n-\frac{2n^4}{\sqrt{4n^6+3n^5+1}}&=\frac{\left(n-\frac{2n^4}{\sqrt{4n^6+3n^5+1}}\right)\left(n+\frac{2n^4}{\sqrt{4n^6+3n^5+1}}\right)}{n+\frac{2n^4}{\sqrt{4n^6+3n^5+1}}}\\
&=\frac{3n^7+n^2}{(4n^6+3n^5+1)\left(n+\frac{2n^4}{\sqrt{4n^6+3n^5+1}}\right)}>\frac{3n^7}{8n^7}
      \end{align*}
      Also, 
      \[
\frac{3n^7+n^2}{(4n^6+3n^5+1)\left(n+\frac{2n^4}{\sqrt{4n^6+3n^5+1}}\right)}<\frac{3n^7+\frac{9}{4}n^6+\frac{3}{4}n}{(4n^6+3n^5+1)\left(n+\frac{2n^4}{\sqrt{4n^6+3n^5+1}}\right)}=\frac{\frac{3}{4}n}{n+\sqrt{\frac{4n^8}{4n^6+o(n^6)}}}
      \]
      Applying squeeze, we see that
      \[
\lim_{n\to\infty}\frac{\frac{3}{4}n}{n+\sqrt{\frac{4n^8}{4n^6+o(n^6)}}} = \frac{3}{8}\;\;\;\text{and}\;\;\;\lim_{n\to\infty}\frac{3n^7}{8n^7}=\frac{3}{8}
      \]
      Therefore, the required limit exists and equals $\frac{3}{8}$.
      \item We recognise that the sequence enumerated by $\left(\frac{3^n+5}{3^n+3}\right)^{3^{n+1}}$ is the subsequence of the sequence enumerated by $\left(\frac{n+5}{n+3}\right)^{3n}=\left(1+\frac{2}{n+3}\right)^{3n}$, whose limit is equal to $\left(1+\frac{2}{n}\right)^{3(n-3)}$, and equal to its subsequence $\left(1+\frac{2}{2n}\right)^{3(2n-3)}$. We then evaluate this subsequence's limit.
      \[
\lim_{n\to\infty}\left(1+\frac{1}{n}\right)^{6n-9}=\frac{\lim_{n\to\infty}\left(1+\frac{1}{n}\right)^{6n}}{\left(1+\frac{1}{n}\right)^9}=\frac{e^6}{1}=e^6
      \]
      \item Use squeeze theorem, for all $n\geq 10$,
    \[
3^4=(3^{4n})^\frac{1}{n}<\left(3^{4n}+\left(4+\frac{1}{n}\right)^{3n}\right)^{\frac{1}{n}}<(2\cdot 3^{4n})^\frac{1}{n}=2^\frac{1}{n}\cdot81
    \]
    And as $\lim_{n\to\infty} 2^\frac{1}{n}=1$, we have the limit exist and equal 81.
    \end{enumerate}
    \item Let $a\in\mathbb{R}$. Take a rational sequence $(x_n)$ and an irrational sequence $(y_n)$ such that $x_n\to a$ and $y_n\to a$. Then
    \[
f(x_n)=\sqrt{4x_n-8}\to \sqrt{4a-8}\;\;\;\text{ and }\;\;\;f(y_n)=y_n-1\to a-1
    \]
If $f$ is continuous at $x=a$, then
\[
\sqrt{4a-8}=\lim_{n\to\infty}f(x_n)=\lim_{n\to\infty}f(y_n)=a-1
\]
so that $a=3$. It follows that if $a\neq 3$, then $f$ is not continuous at $x=a$.\\Next we prove that $f$ is continuous at $x=3$, i.e., $\lim_{x\to 3} f(x)=f(3)=2$.\\
Let $\varepsilon>0$. We choose $\delta=\min\{1,\frac{\varepsilon}{2}\}$. Then, if $|x-3|<\delta$, we have
\[
|f(x)-2|=\begin{cases}
|\sqrt{4x-8}-2|=\frac{|4x-12|}{\sqrt{4x-8}+2}<\frac{4|x-3|}{2}<2\cdot\frac{\varepsilon}{2}=\varepsilon&\text{if }x\text{ is rational}\\
|x-3|<\frac{\varepsilon}{2}<\varepsilon&\text{if }x\text{ is irrational}
\end{cases}
\]
In other words, $|x-3|<\delta\Rightarrow |f(x)-f(3)|<\varepsilon$, so $f$ is continuous at $x=3$.
  \end{enumerate}
  \item\begin{enumerate}%Q2
  \item \begin{enumerate}%Q2a
  \item We recognise this series is eventually non-negative, so apply simplified root test:
  \begin{align*}
\rho &= \lim_{n\to\infty}\left(n^2\left(6+\frac{1}{n}\right)^n\left(1+\frac{1}{3n^2}\right)^{-6n^3}\right)^{\frac{1}{n}}\\
&=\lim_{n\to\infty}n^{\frac{2}{n}}\lim_{n\to\infty}\left(6+\frac{1}{n}\right)\lim_{n\to\infty}\left(1+\frac{1}{3n^2}\right)^{-6n^2}\\
&=1\cdot 6\cdot \lim_{n\to\infty}\left(1+\frac{1}{n}\right)^{-2n}\\
&=6\cdot e^{-2}<1
  \end{align*}
  Therefore, it converges.
  \item As $\sqrt{n^4+5n+1}-n^2=\frac{(\sqrt{n^4+5n+1}-n^2)(\sqrt{n^4+5n+1}+n^2)}{\sqrt{n^4+5n+1}+n^2}=\frac{5n+1}{\sqrt{n^4+5n+1}+n^2}$. Also note that
  \[
\frac{5n+1}{\sqrt{n^4+5n+1}+n^2}>\frac{5n}{2\sqrt{n^4+5n+1}}
  \]
  Apply limit comparison test on series enumerated by $\frac{5n}{2\sqrt{n^4+5n+1}}$ and $\frac{1}{n}$, we have
  \[
\rho = \lim_{n\to\infty}\frac{\frac{5n}{2\sqrt{n^4+5n+1}}}{\frac{1}{n}}=\lim_{n\to\infty}\frac{5}{2}\sqrt{\frac{n^4}{n^4+5n+1}}=\frac{5}{2}>0
  \]
  Therefore, as series enumerated by $\frac{1}{n}$ diverges, $\sum_{n=1}^\infty\frac{5n}{2\sqrt{n^4+5n+1}}$ diverges, and by comparison test, the series in the question diverges.
  \end{enumerate}
  \item $\sum_{i=1}^\infty\frac{\cos(n\pi)}{\sqrt{n+1-\sqrt{n}}}=\sum_{i=1}^\infty\frac{(-1)^n}{\sqrt{n+1-\sqrt{n}}}$. By alternating series test,
  \begin{itemize}
    \item $\frac{1}{\sqrt{n+1-\sqrt{n}}}\geq 0$ for all $n$
    \item $\frac{1}{\sqrt{n+1-\sqrt{n}}}$ is decreasing. It is obvious as the next term has a denominator $\sqrt{n+2-\sqrt{n+1}}$, since both denominator are positive, we can compare their squares: $(n+1-\sqrt{n})-(n+2-\sqrt{n+1})=-1+(\sqrt{n+1}-\sqrt{n})<0$, therefore, the next term's denominator is always larger, which makes the next term smaller than any previous term.
    \item $\lim_{n\to\infty}\frac{1}{\sqrt{n+1-\sqrt{n}}}=0$.
  \end{itemize}
  This series converges conditionally. We then show that this series does not converges absolutely. i.e., $\sum_{n=1}^\infty \frac{1}{\sqrt{n+1-\sqrt{n}}}$ does not converge. We see this by comparing this series against $\sum_{n=1}^\infty \frac{1}{n}$. It is clear that for $n>2$, $\frac{1}{n}<\frac{1}{\sqrt{n+1-\sqrt{n}}}$, and by comparison test, since $\sum_{n=1}^\infty$ diverges, $\sum_{n=1}^\infty \frac{1}{\sqrt{n+1-\sqrt{n}}}$ also diverges.
  \end{enumerate}
  \item \begin{enumerate} \item Yes, $(a_n)$ converges. We claim that $(a_n)$ is monotone decreasing and bounded below by $-6$. From monotone convergence theorem, $(a_n)$ converges. Next we prove the two claim above.\\First $(a_n)$ is bounded below by $-6$. We write
  \[
a_{n+1}=\frac{9a_n}{3-a_n}=-9+\frac{27}{3-a_n}
  \]
  Since $a_1=-5>-6$, $a_2>-9+\frac{27}{3-(-6)}=-6$. And by induction, we can show that $a_n>-6$ for all $n$.\\Next we show that $(a_n)$ is monotone decreasing. We note the obvious fact that $a_n<0$ for all $n$. Then, observe
  \[
a_{n+1}-a_n = \frac{9a_n}{3-a_n}-a_n = \frac{a_n(a_n+6)}{3-a_n}
  \]
  Note, that $a_n<0$, $a_n+6>0$ and $3-a_n>0$, so $a_{n+1}-a_n<0$. This suggests that for all $n$, $a_{n+1}<a_n$. This prove our claim.
  \item Let $M:=\max\{f(x_1),f(x_2),f(x_3)\}$ and $m=\min\{f(x_1),f(x_2),f(x_3)\}$. We see that the right hand side
  \[
m=\frac{4}{\frac{1}{m}+\frac{2}{m}+\frac{1}{m}}\leq \frac{4}{\frac{1}{f(x_1)}+\frac{2}{f(x_2)}+\frac{1}{f(x_3)}}\leq \frac{4}{\frac{1}{M}+\frac{2}{M}+\frac{1}{M}}=M
  \]
  And by intermediate value theorem, we have our $c$. 
  \end{enumerate}
  \item\begin{enumerate}
  \item We note that the required sum, by telescoping, equals
  \[
\lim_{k\to\infty}\sum_{n=1}^k (a_n-2a_{n+1}+a_{n+2})=\lim_{k\to\infty}{a_1-a_2-a_{n+1}+a_{n+2}}=a_1-a_2
  \]
  Therefore, the series converges
  \item
  \begin{enumerate}
    \item Let $x_1,x_2\in[1,2)$. By continuous extension theorem, $f$ is continuous on $[1,2]$. Therefore, for any $\varepsilon>0$, there exists $\delta>0$ such that $|f(x_2)-f(x_1)|<\frac{\varepsilon}{2M}$ where $M=\max_{1\leq x<2}f(x)$ ($0<M <\infty$, the lower bound is specified by the question and upper bound given by extreme value theorem) for all $|x_2-x_1|<\delta$. Next we show $g$ is uniformly continuous on $[1,2)$. For the same  $\delta$ , such that $|x_1-x_2|<\delta$,
    \[
|g(x_1)-g(x_2)|=|f(x_1)^2-f(x_2)^2| = |f(x_1)+f(x_2)||f(x_1)-f(x_2)|<2M\cdot\frac{\varepsilon}{2M}=\varepsilon
    \]
    Therefore, $g$ is uniformly continuous on $[1,2)$.
    \item No. Let $(x_n)$ $(u_n)$ be two series such that $x_n = 2-\frac{1}{n}$ and $u_n = 2-\frac{2}{n}$. It is clear that $x_n-u_n\to 0$. However, $h(x_n)-h(u_n)=\frac{n}{2}\to\infty$.Therefore, $h$ is not uniformly continuous on $[1,2)$.  
  \end{enumerate}
  \end{enumerate}
  \item\begin{enumerate}
  \item For any $\varepsilon>0$, choose $\delta = \min\{\frac{1}{2},\sqrt{\frac{\varepsilon}{6}}\}$ such that $|x-2|<\delta$, which gives $\frac{1}{2}<x-1<\frac{3}{2}$
  \[
\left|\frac{3x^2}{x-1}-12\right| = \left|\frac{3(x-2)^2}{x-1}\right|<2\cdot3\delta^2\leq\varepsilon
  \]
  and the result follows.
  \item From the limit, we have, for any $\varepsilon>0$, there exists $M>0$ such that $n>M$ gives
  \[
|3a_{n+1}-a_n-1|<\varepsilon
  \]
  which is equivalent to 
  \[
\frac{a_n}{3}+\frac{1-\varepsilon}{3}<a_{n+1}<\frac{a_n}{3}+\frac{1+\varepsilon}{3}
  \]
  Recursively apply this inequality until $n=M+1$, we have
  \[
\frac{a_{M+1}}{3^{n-M-1}}+f_1(\varepsilon)<a_{n+1}<\frac{a_{M+1}}{3^{n-M-1}}+f_2(\varepsilon)
  \]
  Taking limit on the inequality and let $\varepsilon\to 0$, we will have, by squeeze, that $a_{n+1}\to 0$, which implies $(a_n)$ converges.
  \end{enumerate}
  \item \begin{enumerate}
  \item We note that $\lim_{x\to4^+}\left[\frac{x}{4}\right]=1$ and $\lim_{x\to4^-}\left[\frac{x}{4}\right]=0$. Also, $\lim_{x\to4^+}\left[\frac{4}{x}\right]=0$ and $\lim_{x\to4^-}\left[\frac{4}{x}\right]=1$. Then 
  \[
\lim_{x\to4^+}\tan\left(\left[\frac{x}{4}\right]+\left[\frac{4}{x}\right]\right)=\lim_{x\to4^+}\tan 1 = \tan 1 =\lim_{x\to4^-} \tan 1 =\lim_{x\to4^-}\tan\left(\left[\frac{x}{4}\right]+\left[\frac{4}{x}\right]\right)
  \]
  Therefore, the limit is $\tan 1$.
  \item Without loss of generality, let $x>y$. Then we have, $|f(x)-f(y)|\geq x-y$ for all $x,y\in\mathbb{R}$. We claim that $f$ is monotone increasing or monotone decreasing. \\We prove this claim by contradiction. Consider the case that $f$ increases on some interval $(a,x_M)$ and decreases on $(x_M,b)$, then $f$ takes value $(f(a),f(x_M))$ on the first interval and $(f(b),f(x_M))$ on the second. We can always choose a $\lambda_1\in (a,x_M)$ and $\lambda_2\in(x_M,b)$ such that $\max\{f(a),f(b)\}<f(\lambda_1)=f(\lambda_2)<f(x_M)$, but it contradicts the assumption. Similarly, the other possibility will also lead to contradiction. \\ We further note that $f$ cannot be constant for any interval, as it will lead to the contradiction also. Since $f$ is strictly monotone, there exists an inverse $f^{-1}:=g$ such that $f(g(x))=x$. 
  \end{enumerate} 
\end{enumerate}
\end{document}